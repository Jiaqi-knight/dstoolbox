\documentclass[11pt]{report}
\usepackage{Preamble} % see file Preamble.sty

\begin{document}

\begin{titlepage}
    \newgeometry{margin=3cm}
	\centering
    \includegraphics[width=0.5\linewidth]{Images/epfl.png}\\[0.25cm] 	% University Logo
    \textsc{\LARGE École Polytechnique Fédérale de Lausanne}\\ \vspace{\fill}
    \textbf{\textsc{\fontsize{50}{50}\selectfont Report}}\\ \vspace{\fill}		
	\textsc{\LARGE EPFL Master Thesis}\\[0.4cm]
	\rule{\linewidth}{0.2 mm} \\[0.5 cm]
	Lucas Schneeberger \\[2cm] \today
\end{titlepage}
\restoregeometry

\thispagestyle{numberonly}

\begin{summary}
\section*{Abstract}
Helicopters and wind turbines promote an unsteady flow around the blades due to their rotary motion. Dynamic stall models help the designers predicting the subsequent unsteady aerodynamic loads. In this work, Beddoes-Leishman 3rd generation model is applied to ramp-up motions of a flat plate airfoil at moderate Renyolds number. The results are compared to the dynamic stall model designed by Sheng et al specially for low speeds ($M<0.3$). A new stall criterion that retakes Sheng et al. formulation is introduced with extended pitch rate validity range. 
\end{summary}

\tableofcontents

\chapter{Introduction}
\subfile{1-Introduction}

\chapter{Methods}
\subfile{2-Methods}

\chapter{Results}
\subfile{3-Results}

\chapter{Discussion}
\subfile{4-Discussion}

\chapter{Conclusion}
\subfile{5-Conclusion}

\begin{multicols}{2}[\chapter{Nomenclature}]
\subfile{6-Nomenclature}
\end{multicols}

\bibliography{bibliography} 
\bibliographystyle{new-aiaa}

\chapter{Annexes}
\subfile{7-Annexes}

\end{document}
