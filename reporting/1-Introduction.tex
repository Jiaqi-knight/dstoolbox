\section{Context}
For the last 70 years, dynamic stall has been a complex unsolved theoretical problem of fluid dynamics that have constantly aroused interest of scientists and engineers. From early  McCroskey \cite{mccroskey_dynamic_1976} in the 1970s was a pioneer in the field with his phenomenological descriptions. The helicopter aerodynamicists T.S. Beddoes made a lot of progress to model the phenomenon and to understand the underlying physics. Beddoes created the first generation of model in \cite{beddoes_synthesis_1976}. This model, based on two time constants, was largely based on observation and experience. He then presented a second generation of his model in \cite{beddoes_representation_1983}. Together with Leishman, they finally presented in 1989 the third generation of his dynamic stall model, the one often referred as \textit{Leisman-Beddoes model}\footnote{Even though the name in reversed order (\textit{Beddoes-Leishman}) might occur at least as frequently , in the present work this model will be referred as \textit{Leishman-Beddoes model} so that the abreviation LB cannot be confused with the abbreviation for \textit{boundary layer}, BL.} nowadays. The  lecture of the original paper \cite{leishman_semi-empirical_1989} is however not recommended, as the nomenclature is confusing for a correct implementation. Leishman also proposed an implementation of the third-generation model in \cite{leishman_state-space_1989} with the value of the time-constants for three different airfoils. The third-generation model provided a solid way of predicting the lift, drag and pitching moment characteristics of an airfoil in high speed (M $>0.3$) and high Reynolds number (Re $> 10^6)$) conditions. Since the main field of application at that time was the helicopter aerodynamics, the airfoil shape chosen was often the symmetric  NACA0012 describing a pitching or plunging motion in time. 

Dynamic stall  benefited from a renewed interest since the 2000s due to the development of wind energy, especially for vertical axis wind turbines, the VAWTs. In this context, Sheng et al. have started to look at a dynamic stall criterion that is more adequate to lower speeds (M $<0.3$). This criterion was further developed in \cite{sheng_modified_2008} and gave birth to a modified version of the third-generation LB model. At the same time, the focus shifted from pitching motions to ramp-up motions. 


Recent developments from \cite{tank_possibility_2017} showed that dynamic stall models still deserve further investigation of the static aerodynamic properties at moderate Reynolds number

\section{Motivation}

In \cite{sheng_modified_2008}, Sheng et al. approximate the dynamic stall angle as a piecewise linear function of the reduced pitch rate. At the same time, the choice of the critical angle of attack for stall onset assessment seemed rather arbitrary. For these reasons, there seemed to be room for improvement in this stall onset criterion and the associated modified LB model. 
Indeed, it was already observed in the data exploited in the present study that the dynamic stall time was an exponential function of the reduced pitch rate. This observation lead to the hypothesis that the piecewise linear fit of Sheng et al. was only an approximation of what should naturally be an exponential function. Additionally, the critical angle should be retought as of an angle bearing more physical meaning than the one proposed by Sheng in \cite{sheng_modified_2008}.
\section{Goals}

The present study aims at optimizing the compromise between accuracy and lightness that has been reached in the established dynamic stall models. Special attention has been dedicated to Beddoes-Leishman third generation model, to the modifications proposed by Sheng 2008 and to Goman-Khrabrov model. 

\begin{wrapfigure}{r}{5cm} %here, bottom of page, top of page, empty page in order of preference
    \centering
    \includegraphics[width=5cm]{Images/epfl.png}
    \caption{Check out these nice centered captions though}
    \label{fig:my_label}
\end{wrapfigure}

Check out referencing, use \texttt{parencite} for authoryear style \parencite{reflabel}, or just \texttt{cite} if numeric style set in preamble.

\begin{invsummary}
Check out the custom header
\end{invsummary}