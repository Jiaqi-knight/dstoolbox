For the past 70 years, dynamic stall has constantly aroused interest of scientists and engineers. 
It is one of the complex theoretical problem of fluid dynamics that remain unsolved.
From as early as the 1970s, pioneers such as McCroskey \cite{mccroskey_dynamic_1976} established precise phenomenological descriptions of the of the different stages of dynamic stall and its associated vorticity formation. The helicopter aerodynamicist T.S. Beddoes made much progress modelling the dynamic stall phenomenon and understanding its underlying physics \cite{beddoes_synthesis_1976, beddoes_representation_1983}. 

Beddoes came up with the 1st generation of his dynamic stall model in 1976 \cite{beddoes_synthesis_1976}.
Based on both his own observation and Ashley's theoretical framework for aeroelasticity \cite{ashley_piston_1956}, this model first introduced the indicial response method and the notion of effective angle of attack. These concepts were meant to persist up to the most modern versions of the Leishman-Beddoes (LB) model. 
In the second iteration of his dynamic stall model, Beddoes first gave an approach to trailing edge separation using Kirchhoff model \cite{beddoes_representation_1983}. 
Together with Leishman, they finally presented in 1989 the 3rd-generation \cite{leishman_semi-empirical_1989} dynamic stall model, which is nowadays referred to as \textit{Leishman-Beddoes model}\footnote{Even though the name in reversed order (\textit{Beddoes-Leishman}) might occur at least as frequently, in the present work this model will be referred to as \textit{Leishman-Beddoes model} so that the abreviation LB cannot be confused with the abbreviation for \textit{boundary layer}, BL.}.
Influenced by the convenient framework of control theory, Leishman also proposed an alternative formulation using the state-space form of the 3rd-generation model \cite{leishman_state-space_1989}.
Despites the high number of parameters and the empirical origin of some of its parts, the 3rd-generation model provided a solid way of predicting the lift, drag, and pitching moment characteristics of an airfoil in high speed and high Reynolds number conditions (M $>0.3$, Re $> 10^6$). These operating conditions perfectly suited the main field of application at that time: helicopter aerodynamics. For the same reason, the airfoil shape chosen was often the symmetric NACA0012 describing a pitching or plunging motion \cite{beddoes_representation_1983,leishman_semi-empirical_1989,sheng_new_2006,sheng_improved_2007, tank_possibility_2017,rocchio_simple_2020}. 

Dynamic stall received a renewed interest since the 2000s due to the development of wind energy, especially for vertical axis wind turbines (VAWTs). In wind turbine applications, the typical airfoil shapes, motions, and flow conditions differ largely from helicopters. Sheng et al. developed a dynamic stall criterion more adequate to lower speeds (M $<0.3$). This criterion was further developed \cite{sheng_improved_2007} and gave birth to a modified version of the 3rd-generation LB model \cite{sheng_modified_2008}. Simultaneously, researchers extended their focus to ramp-up motions, which are better suited than pitching motions for isolating the influence of the pitch rate. 

Thanks to the advent of particle image velocimetry, additional phenomenological description of the leading edge vortex formation, a typical feature of dynamic stall, was provided by Mulleners \& Raffel \cite{mulleners_dynamic_2013}. 

Recent developments of Tank et al. \cite{tank_possibility_2017} showed that dynamic stall models still require further investigation of the airfoils static aerodynamic properties at moderate Reynolds number. Indeed, static properties are the base every dynamic stall model is built on. Without robust and trustworthy static lift, drag and pitching moment polars, unsteady aerodynamic loads cannot be accurately predicted. 

% Motivation

The present study aims at optimizing the compromise between accuracy, lightness, and physicality that has been reached in the established dynamic stall models. 
The 3rd-generation LB model predicts reasonably the dynamic loads on an airfoil, but it depends on a dozen of tuning parameters. 
Tuning such a large number of parameters can be impractical. This is even more true in the case of a new airfoil, for which few experimental data is available or in new conditions (Reynolds and Mach numbers). The number of required parameter has already been disminished by Sheng et al. \cite{sheng_modified_2008}. Special attention has therefore been dedicated to improving their model, which is also better adapted to lower velocities.  

Sheng et al. \cite{sheng_improved_2007} connected the value of one involved time constant to the parameters of the mathematical dependency between the dynamic stall angle and the reduced pitch rate. 
The nature of this dependency has been found by them to be piecewise linear. 
However, a different kind of relation was found by analizing the collection of ramp-up motions presented in this work: the dynamic stall angle looks more like an exponential function of the reduced pitch rate. 
This observation lead to the hypothesis that the piecewise linear fit of Sheng et al. was only an approximation of what should naturally be an exponential function. 

A second critic can be made to Sheng's model. The critical angle $\alpha_{crit}$ plays a central role, but its definition seems rather arbitrary and lacks of physical meaning. A more physical definition is proposed here. 

Based on these two reproaches to Sheng's stall onset criterion and the associated modified LB model, there seemed to be room for improvement.


 


