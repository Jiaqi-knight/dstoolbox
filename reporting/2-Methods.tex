In this chapter, the 3rd-generation LB model \cite{leishman_semi-empirical_1989} and some of its variations based on contributions from Sheng \cite{sheng_modified_2008} and Bangga \cite{bangga_improved_2020} are presented. An innovative variation of Sheng's model is also introduced. 

In the second section, the experimental setup, some important assumptions about the experimental dynamic stall onset and the experimental data treatement will be explained.

\section{Dynamic stall modeling}
\label{section:ds_model}
Any successful dynamic stall model consists of two things: a trustworthy prediction of the static airfoil characteristics and a correction due to the unsteadiness of the flow. These two parts define the structure of this chapter. 

\subsection{Modelling of the static charateristics}
\label{section:kirchhoff}

% see p. 10 of Beddoes 1983
Beddoes introduced the so-called \textit{Kirchhoff flow approximation}\footnote{This model is sometimes also called \textit{Kirchhoff-Helmholtz model}} \cite{beddoes_representation_1983}. This model gave a relationship for the angle of attack, the separation point and the normal coefficient. The boundary layer separation point, according to Beddoes \cite{beddoes_representation_1983}, can also be assimilated to the forward progress of adverse pressure gradients. In this case, $f$ represents the location of no wall shear stress \footnote{Some authors \cite{serra_exact_2018} disagree with this hypothesis for unsteady flow but we'll still keep it, admitting it is only a hypothesis.}.
The separation (or no shear stress) point is described as a function of the angle of attack $\alpha$ by the following expression : 

\begin{equation}
	f(\alpha) = 
		\begin{cases}
		1-(1-f_{ss})\exp(\frac{\alpha-\alpha_{ss}}{S_1}), &\quad \alpha \leq \alpha_{ss}\\
		f_{\infty}+(f_{ss}-f_{\infty})\exp(\frac{\alpha_{ss}-\alpha}{S_2}). &\quad \alpha > \alpha_{ss}\\
		\end{cases}
	\label{eq:seppoint}
\end{equation}

% about the meaning of f_\infty and f_ss
The form of this function is worth some considerations. It is a piecewise defined function, starting at $f=1$ for $\alpha=\ang{0}$ and with the transition between the two parts happening by construction at $\alpha=\alpha_{ss}$, where $f$ takes the value $f_{ss}$.
When $\alpha \rightarrow \infty$, the function admits a horizontal asymptot at $f_{\infty}$.

% about the meaning of S1 and S2
The constants $S_1$ and $S_2$ in \eqref{eq:seppoint} are optimization parameters that characterize the abruptness of the stall, through controlling the rate of the negative exponential increase of the first part, and the positive exponential decay of the second part of the separation curve. 

The approximation for the normal coefficient as a function of the boundary layer separation location and $\alpha$ is the following: 

\begin{equation}
	C_N (\alpha) = C_ {N_{\alpha}} \left(\frac{1+\sqrt{f}}{2}\right)^2 \alpha
	\label{eq:kirchhoff}
\end{equation}

\noindent where $C_{N_\alpha}$ is the static lift curve slope in the pre-stall regime. By comparing Beddoes \cite{beddoes_representation_1983} and Leishman \& Beddoes \cite{leishman_semi-empirical_1989}, it can be seen that this expression is used indifferently for $C_N$ and $C_L$.

% about the value of S1 and S2
In Eq. \eqref{eq:seppoint}, once $f_{ss}$ and $f_\infty$ are chosen, $S_1$ and $S_2$ are optimized in a least-squares sense, so that the difference between the experimental normal coefficient and the one given by the Kirchhoff model (combining Eq. \eqref{eq:seppoint} \& \eqref{eq:kirchhoff}) is minimal.

Originally, $\alpha_{ss}$ in Eq. \ref{eq:seppoint} is denoted by $\alpha_1$ and is added to the parameters to be optimized. However, $\alpha_1$ is often found to be very close to the static stall angle $\alpha_{ss}$, as mentioned by Leishman \& Beddoes \cite{leishman_semi-empirical_1989}. As a result, it was replaced by the experimentally observed static stall angle $\alpha_{ss}$. 

This model can sometimes be problematic, as discussed later.

\iffalse
\begin{tikzpicture}[
	nonterminal/.style={
		% The shape:
		rectangle,
		% The size:
		minimumsize=6mm,
		% The border: 
		very thick,
		draw=black,
		% The filling:
		color=white,						
	}]
	\node [nonterminal] {\alpha}
\end{tikzpicture}
\fi

\subsection{Leishman-Beddoes model}

The original Leishman-Beddoes is best explained in \cite{bangga_improved_2020}. However, for the convenience of referring to equation numbers in the following sections, a brief overview is provided in this one. 

\subsubsection{Attached flow behavior}

The attached regime is defined as the angles of attack for which the boundary layer follows the airfoil curvature without exhibiting separation, typically between -10 and 10 degrees. The force under this conditions is a linear function of the angle of attack and is the sum of two components, the circulatory and the impulsive lift: 

\begin{equation}
C_N = C_N^C + C_N^I
\end{equation}

\noindent where the circulatory force $C_N^C$ is related to the added circulation due to a sudden deflection. This 

\begin{equation}
C_N^C = C_{N_\alpha} \alpha_{E}
\label{eq:circulatory}
\end{equation}

\noindent where $C_{N_\alpha}$ is the slope of the $C_N(\alpha)$ polar curve in the pre-stall regime and with the \textit{attached effective angle of attack} $\alpha_E = \alpha - X - Y$, where $X$, $Y$ are deficiency functions obtained with a mid-point approximation of the Duhamel's integral \cite{leishman_principles_2006}:

\begin{eqnarray}
X(n)= & X(n-1) \exp \left(-b_{1} \beta^{2} \Delta S\right)+A_{1} \Delta \alpha \exp \left(-b_{1} \beta^{2} \Delta S / 2\right) \\
Y(n)= & Y(n-1) \exp \left(-b_{2} \beta^{2} \Delta S\right)+A_{2} \Delta \alpha \exp \left(-b_{2} \beta^{2} \Delta S / 2\right)
\end{eqnarray}

\begin{equation}
C_{N}^{I}=\frac{4 K_{\alpha} T_{l}}{M}\left(\frac{\Delta \alpha}{\Delta t}-D\right)
\label{eq:impulsive}
\end{equation}

 The impulsive term corresponds to the pressure difference caused by the mass displacement. It is also called \textit{added mass} in an aeroelasticity context.

\subsubsection{Leading edge separation}

According to Niven \& Galbraith \cite{niven_modelling_1997}, stall onset can occur in pure subsonic conditions either by leading edge separation or trailing edge separation. Any general model for dynamic stall must therefore include both. 

Leading edge separation is known to occur when the pressure at leading edge goes down to a critical pressure, depending on the Mach number (see Evans \& Mort correlation in \cite{sheng_new_2006}). Additionally, in \cite{leishman_semi-empirical_1989}, Leishman \& Beddoes make the assumption the pressure distribution is directly related to the normal force coefficient. Simultaneously, the unsteadiness of the flow induces a delay in the boundary layer and therefore pressure gradients are delayed with respect to the steady case. Therefore, a first-order lag is applied to the normal coefficient to define $C_N^{\prime}$. The transfer function from $C_N$ to $C_N'$ is: 

\begin{equation}
\frac{C_N^{\prime}(s)}{C_N(s)} = \frac{1}{1+T_p s}
\label{eq:cnprime_laplace}  
\end{equation}

The above equation is in the Laplace domain. A numerical solution in the time domain is given in Section \ref{section:duhamel}. $C_N^{\prime}$ controls the onset of stall by being a direct measure of the pressure distribution on the airfoil and is used to define the \textit{separated effective angle of attack} $\alpha_f$ in the following manner:

\begin{equation}
\alpha_f = \frac{C_N^{\prime}}{C_{N_\alpha}}
\label{eq:separated_effective_alpha}
\end{equation}

\subsubsection{Trailing edge separation}

As the angle of attack increases, stall is approached. As already presented in Section \ref{section:kirchhoff}, stall can occur through progressive separation of the boundary layer from the trailing edge forwards, depending on the airfoil shape. The delayed separation point $f^{\prime}$ is then computed from the efficient angle of attack $\alpha_f$, by reusing the optimal stall abruptness parameters $S_1$ and $S_2$ from Equation \eqref{eq:seppoint}.  The relationship between $\alpha_f$ and $f'$ is assumed to be the same as the one between $\alpha$ and $f$ in the static case:

\begin{equation}
f^{\prime}(\alpha)=\left\{\begin{array}{ll}
1-(1-f_{ss}) \exp \left(\frac{\alpha_{f}-\alpha_{ss}}{S_{1}}\right), & \alpha_{f} \leq \alpha_{ss} \\
f_\infty + (f_{ss}-f_\infty) \exp \left(\frac{\alpha_{ss}-\alpha_{f}}{S_{2}}\right), & \alpha_{f}>\alpha_{ss}
\end{array}\right.
\end{equation}

A first order lag of magnitude $T_f$ is then applied to the effective separation point $f^{\prime}$ to compute the final value of the unsteady separation point, $f^{\prime \prime}$. Like in Eq. \eqref{eq:cnprime_laplace}, the transfer function from $f'$ to $f''$ writes:

\begin{equation}
\frac{f^{\prime \prime}(s)}{f^{\prime} (s)} = \frac{1}{1 + T_f s}
\end{equation}

\noindent and the modified Kirchhoff normal coefficient $C_N^f$ is computed using Kirchhoff relationship \eqref{eq:kirchhoff} applied to $f^{\prime \prime}$ for the circulatory part, summed with the impulsive contribution.

\begin{equation}
	C_{N}^{f}=C_{N_{\alpha}}\left(\frac{1+\sqrt{f^{\prime \prime}}}{2}\right)^{2} \alpha_{E}+C_{N}^{I}
	\label{eq:mod_kirchhoff}
\end{equation}

\subsubsection{Dynamic stall and vortex shedding}

The phenomenon of dynamic stall is well known for inducing a leading edge vortex (LEV) \cite{mccroskey_dynamic_1976,mulleners_onset_2010,mulleners_dynamic_2013}. The LEV participates in increasing the lift until a certain time $\Delta t_{ds}=t_{ds}-t_{ss}$ after exceeding of the static stall angle. Right after this time delay, the vortex is shed into the wake. The vortex lift follows the implementation of Bangga et al. \cite{bangga_improved_2020}:

\begin{equation}
	C_{N}^{v}(n)=
	\begin{cases}
		C_{N}^{v}(n-1) \exp \left(-\frac{\Delta t_c}{T_{v}}\right)+\left[C_{v}(n)-C_{v}(n-1)\right] \exp \left(-\frac{\Delta t_c}{2 T_{v}}\right), & \text{if } 0<\tau_v<T_{vl} \\
		C_{N}^{v}(n-1) \exp \left(-\frac{\Delta t_c}{T_{v}}\right), & \text{otherwise} \\
	\end{cases}	
	\label{eq:vortex_lift}
\end{equation}

\noindent where $T_{vl}$ is the adimensional time at which the LEV reaches the trailing ede \cite{leishman_semi-empirical_1989}, and $\tau_v$ is the adimensional vortex time, defined as follows: 

\begin{equation}
	\tau_{v}(n)=\left\{\begin{array}{ll}
		\tau_{v}(n-1)+0.45 \frac{\Delta t}{c} V, & \text { if } \quad C_N^{\prime}>C_{N}^{CRIT} \\
		0, & \text { if } \quad C_N^{\prime}<C_{N}^{CRIT} \quad \text { and } \quad \Delta \alpha_{n} \geq 0 \\
		\tau_{v}(n-1), & \text { otherwise }
		\end{array}\right.
	\label{eq:adim_vortex_time}	
\end{equation}

\noindent where $C_N^{CRIT}=C_N(\alpha_{ss})$, the normal coefficient at stall in the static case. An analogy can be made between the vortex lift and a leaking tank. The vortex lift is represented by the level of the tank, which empties itself continuously over time. The tank starts empty. When stall is attained ($C_N>C_N^{crit}$), the tank starts filling up (the vortex lift $C_N^v$ rises), and a counter is started ($\tau_V$). When $tau_v$ reaches $T_{vl}$, the tank intake is stopped and empties itself back to the initial state. 

\subsection{Sheng criterion}
\label{section:sheng_criterion}
% TODO: rewrite with reader in mind
Sheng et al. observe \cite{sheng_new_2006} that the dynamic stall criterion introduced by Leishman \& Beddoes does not work well for low Mach numbers ($M<0.3$). They instead define a new stall criterion based 
on a lagged angle of attack, $\alpha'$. They first establish a linear relationship between the dynamic stall angle $\alpha_{ds}$ and the reduced pitch rate $r$. 

\begin{equation}
	\alpha_{ds}(r) = T_\alpha r+\alpha_{ds,0}
	\label{eq:linfit}
\end{equation}

\noindent and then define the lagged angle of attack $\alpha'$ as:

\begin{equation}
\Delta \alpha' = \Delta \alpha\left[1-e^{-t_c/T_\alpha} \right]
\label{eq:alpha_lag}
\end{equation}

\noindent with $t_c=t \cdot c/U_{\infty}$, the convective time and $T_\alpha$ the slope of the linear fit in Equation \ref{eq:linfit}. The stall criterion for an airfoil in ramp-up motion is therefore: 

\begin{equation}
\alpha' > \alpha_{crit}
\label{eq:stall_criterion}
\end{equation}

In Sheng's article, $\alpha_{crit}$ varies with $r$. 

\begin{equation}
\alpha_{crit} =
\begin{cases}
\alpha_{ds0}, &\quad \text{if} \quad r \geq r_0 \\
\alpha_{ss} + (\alpha_{ds0}-\alpha_{ss}) \frac{r}{r_0}, &\quad \text{if} \quad r < r_0 \\
\end{cases}
\label{eq:alpha_crit}
\end{equation}

\noindent with $r_0$ generally around 0.01. In order to integrate this criterion in the LB dynamic stall model, $\alpha'$ is used a replacement for $\alpha_f$ to define the delayed separation point $f^{\prime}$. As in Leishman \& Beddoes \cite{leishman_semi-empirical_1989}, $f^{\prime \prime}$ is then computed from $f^{\prime}$ and $C_N^f$ is found using Kirchhoff model, using Equation \eqref{eq:mod_kirchhoff}. 

\subsection{Modified Sheng criterion}
\label{section:expfit}


% TODO: Put an introductive sentence here. 
% TODO: A time-based approach would lead to a linear decay of delta_t_ds with pitch rate. This is not what we are basing upon here ...
The time at static stall angle $t_{ss}$ is defined as the time $t$ at which the airfoil reaches the static stall angle $\alpha_{ss}$.

\begin{equation}
\alpha_{ss} = \alpha(t_{ss})
\end{equation}

\begin{equation}
	\alpha_{ds} = \alpha(t_{ds})
\end{equation}

\subsubsection{Exponential fit of the dynamic stall angle}

Since observations from Mulleners \& Raffel \cite{mulleners_onset_2010,mulleners_onset_2012,mulleners_dynamic_2013} and Sheng et al. \cite{sheng_new_2006,sheng_modified_2008} were restricted to $r<0.05$, it appears necessary to investigate what happens for higher pitch rates. From experimental observation, it has been noticed that the dynamic stall angle tends to reach a plateau for $r>0.05$, as shown in Figure  \ref{fig:expfit}. This lead to the modification of Sheng criterion with an exponential fit for $\alpha_{ds}(r)$.

Based on this observation, we use an exponential fit to identify the dependency between the dynamic stall angle $\alpha_{ds}$ and $r$ of the form: 

\begin{equation}
\alpha_{ds}(r) = A-(A-\alpha_{ss})e^{-Br} = A(1-e^{-Br})+\alpha_{ss}e^{-Br}
\label{eq:alpha_ds_r}
\end{equation}

\noindent where the value of $A$ defines the plateau ($r \rightarrow \infty$). The formulation is consistant with the static observation, since the limit as $r \rightarrow 0$ (static case) is equal to $\alpha_{ss}$. The variable $B$ is the rate of increase in between these two limits. This observation must now be adapted to Sheng's formulation to integrate it in a modified LB model. In order to do so, the remainin remaining challenge is to find a way to express $T_\alpha$ as a function of $r$ from these two coefficients $A$ and $B$.

\subsubsection{Expression for the pitch angle delay constant}
As the slope of $\alpha_{ds}(r)$ is not anymore constant, $T_\alpha$ has to be allowed to vary with $r$. To find a suitable equation to compute the necessary $T_{\alpha}$, the response to a ramp imput $\alpha(t) = \alphadot(t-t_0)$ is investigated. 
We know from Section \ref{section:ramp_input} that the lagged angle attack will be of the form:

\begin{equation}
\alpha'(t) = \alphadot\left[t- \tau(1-e^{-t/ \tau})\right]
\end{equation}

\noindent $\tau$ being the dimensional equivalent of $T_\alpha$ in time domain. In other words, $T_\alpha = \tau c/(2V)$.

Evaluating this function at $t=t_{ds}$ and choosing $\alpha_{crit}=\alpha_{ss}$ in Equation \eqref{eq:stall_criterion}, we obtain : 

\begin{equation}
\alpha'(t_{ds}) = \alphadot\left[t_{ds} - \tau(1-e^{-t_{ds} / \tau})\right] = \alpha_{ss}
\label{eq:alpha_ds_tau}
\end{equation}

If the static properties of the airfoil and the time of dynamic stall $t_{ds}$ are known, this equation can be solved with a numerical solver for $\tau$, as shown on Figure \ref{fig:Talpha_r}.

\begin{figure}[h]
	\centering
	\includegraphics[width=\textwidth]{Sheng/Talpha_r.png}
	\caption{$T_\alpha$ as a function of $r$}
	\label{fig:Talpha_r}
\end{figure}

\subsubsection{Algorithm for pitch-rate-dependent time constant}

\begin{enumerate}
\item An airfoil with chord $c$ and known static stall angle $\alpha_{ss}$ is chosen for the experiment. 
\item The ramp-up motion is started with a defined $\alphadot$ and $r$.
\item The stall onset angle is predicted using $r$ and Equation \eqref{eq:alpha_ds_r}. From there the time of dynamic stall $t_{ds}$ is predicted using $\alphadot$.
\item $T_\alpha$ is computed by solving Equation  \eqref{eq:alpha_ds_tau} for $\tau$. $\alpha'$ is computed in real time using this result. 
\item When $\alpha' > \alpha_{crit}$, the stall criterion is attained and dynamic stall can be considered to have started. 
\end{enumerate}

\subsubsection{Modified Leishman-Beddoes}
\label{section:Sheng-LB}

As already mentioned in Section \ref{section:sheng_criterion}, once $\alpha'$ has been obtained through the above-described procedure, it can be used to define the delayed separation point $f'$ in LB model as follows: 

\begin{equation}
	f^{\prime}(\alpha)=\left\{\begin{array}{ll}
		1-(1-f_{ss}) \exp \left(\frac{\alpha^{\prime}-\alpha_{crit}}{S_{1}}\right), & \alpha^{\prime} \leq \alpha_{ss} \\
		f_\infty + (f_{ss}-f_\infty) \exp \left(\frac{\alpha_{crit}-\alpha^{\prime}}{S_{2}}\right), & \alpha^{\prime}>\alpha_{ss}
		\end{array}\right.
		\label{eq:fp_sheng}
\end{equation}

\noindent where all coefficients are retaken from the static case \eqref{eq:seppoint}. It is also worth remembering that $\alpha_{crit}$ is defined by Eq. \eqref{eq:alpha_crit} in Sheng-BL model and simply by $\alpha_{crit} = \alpha_{ss}$ in the new version with exponential fit. 

Moreover, to compute the adimensional vortex time, $C_N^{\prime}$ cannot be used in Equation \eqref{eq:adim_vortex_time} to assess stalled state, as $C_N'$ does not exist in this model. It is then replaced with the new stall criterion, $\alpha'>\alpha_{crit}$:

\begin{equation}
	\tau_{v}(n)=\left\{\begin{array}{ll}
		\tau_{v}(n-1)+0.45 \frac{\Delta t}{c} V, & \text { if } \quad \alpha'>\alpha_{crit}\\
		0, & \text { if } \quad \alpha'<\alpha_{crit} \quad \text { and } \quad \Delta \alpha_{n} \geq 0 \\
		\tau_{v}(n-1), & \text { otherwise }
		\end{array}\right.
	\label{eq:adim_vortex_time_sheng}	
\end{equation}

The double-delayed separation location $f''$ is computed from $f'$, originally by applying a time delay $T_v$.  $T_v$ is related to vortex growth and was therefore intepreted to be a typing error\footnote{$T_f$ is not used in the model despite being mentioned in the nomenclature of Sheng's article \cite{sheng_modified_2008}.}. The use of $T_f$ makes more sense to the author in that case: 

\begin{equation}
	\Delta f''(t_c) = \Delta f'(t_c)(1-e^{t_c/T_f})
	\label{eq:fpp_sheng}
\end{equation}



\section{Experimental setup}
\label{section:exp_setup}
The experimental data consists of a collection of ramp-up experiments, during which the angle of attack, the normal force $N$, and the chord-wise force $C$ where recorded using a load cell. They are non-dimensionalized by the dynamic pressure times the airfoil's planform $A$. The normal and chordwise force coefficients are computed: 

\begin{eqnarray}
	C_N =& \frac{N}{1/2 \rho V^2 A} \\
	C_C =& \frac{C}{1/2 \rho V^2 A} \\
	\label{eq:force_coeffs}
\end{eqnarray}

\noindent where $c=0.15\si{\meter}$ is the chord length of the flat plate airfoil. 

\subsection{Static characteristics}

The dynamic stall properties are best defined as the sum of static stall properties and a correction due to the unsteady motion. With this approach, it becomes evident that to have a accurate prediction for the dynamic loads, the static data must be reliable. 


The static properties of the modified flat plate airfoil were determined experimentally in the SHARX towing tank at EPFL, similar to the one described by Henne et al. \cite{henne_dynamic_2018}. The airfoil is a flat plate with chord length of $c=0.15 \si{\meter}$, span $b=0.6 \si{\meter}$, and thickness $h=0.02 \si{\meter}$. The flat plate has a rounded trailing edge and its leading edge is sharpened with a $\beta = \ang{45}$ bevel on both side, as shown on Figure \ref{fig:flatplate}.

\begin{figure}[h]
	\centering
	\includegraphics[width=5cm]{flatplate.png}
	\caption{Transverse cut of the flat plate airfoil}
	\label{fig:flatplate}
\end{figure}

The load cell was calibrated with no inflow velocity to compensate for the load cell offset, inertial and  buoyancy forces. The calibration test was run for $10 \si{\second}$ and the signal was averaged to a single calibration value for each axis. This value is then subtracted from the signals during experiments. The calibration procedure is repeated every day so that the calibration values are valid only on the same day, as external perturbations such as atmospheric pressure might change from one day to another. 

After calibration was done, the static loads were measured at $Re=\num{8.4e4}$. The measurements were taken from an angle of attack of $\ang{-5}$ to $\ang{30}$ with a step of $\ang{1}$ in between measurement point. The signal coming from the load cell was average for 10s in order to yield a single force at each pitch position. The series was repeated five times and the normal coefficient value was then averaged to a single one for each angle of attack. 

\begin{figure}[h]
	\centering
	\includegraphics[width=10cm]{static_flatplate.png}
	\caption{Normal coefficient curve for the modified flat plate airfoil, in steady-state from angles of attack from $\alpha=\ang{-5}$ to $\ang{30}$}
	\label{fig:static_flatplate}
\end{figure}

\subsection{Dynamic stall experiments}

Since the beginning of his dynamic stall model, Beddoes adopted an indicial formulation, meaning that the model is defined by the response to a sudden input rise from 0 to 1, a step input. Then, for a linear system, any input can be considered as a superposition of step inputs and whose contributions to the overall output can be summed. Thanks to this method, the model is suitable for any arbitrary time evolution of the angle of attack.

Most early validation experiments on dynamic stall consider airfoils undergoing a pitching motion, where the evolution of the angle of attack in time is a sinusoidal curve and the pitch axis is at the quarter chord. 
Pitching motions are an easy way to approximate the trajectory that a section of helicopter blade follows during forward flight \cite{mccroskey_dynamic_1972,mulleners_coherent_2010}. 
However, when the pitching velocity is constant, the influence of the unsteadiness is easier to asses. For that reason, the present study uses ramp-up motions, i.e. $\alpha(t)$ is a ramp.

\begin{figure}[h]
	\centering
	\includegraphics[width=6cm]{ramp_example.png}
	\caption{Time evolution of the experimental angle of attack for $\alphadot=10\si{\degree \per \second}$, $r=0.026$}
	\label{fig:ramp_example}
\end{figure}

The real experimental pitch angle (Figure \ref{fig:ramp_example}, solid line) differs somewhat from an ideal ramp (dashed line). The experimental ramp does not necessarily start at $t=0$ and it has to stop rising at some point. The maximum angle in all the conducted ramp-up experiments was $\alpha_{max}=\ang{30}$. The black diamond marks marks the observed dynamic stall angle, which can be defined emplying different methods depending on the available sensors.  They fit in two categories: either stall onset is assessed based on the force signals \cite{sheng_new_2006} or using visualization techniques \cite{mulleners_coherent_2010}.
Multiples ways of defining stall onset based on the force signals for the case of a NACA 0012 airfoil were shown by Sheng et al. \cite{sheng_new_2006} to be interchangable . These methods consider as equivalent the following symptoms of dynamic stall: 

\begin{enumerate}
	\item deviation of $C_N$ from its attached-flow slope
	\item $C_M$ dropping by 0.05 
	\item minimum value of the chord-wise force \footnote{Sheng et al. use the maximum, but they define the chord-wise force as being positive in a direction opposed to the flow direction at zero angle of attack. Here we use the opposite convention.}
	\item deviation of $C_D$ from its attached-flow slope
	\item deviation of $C_p$ about the quarter-chord
	\item drop of $C_p$ at the leading edge
\end{enumerate}

We make the same assumption here. The $C_N$ slope does not always visibly increase before the drop due to vortex shedding (Figure \ref{fig:CN_ramp_example}): the angle at maximum $C_N$ is then retained as the dynamic stall angle \cite{mulleners_dynamic_2013}. In the present study, two features were retained for their simplicity of implementation. They are used indifferently as criteria for experimental stall onset:

\begin{enumerate}
	\item a maximum in lift
	\item a minimum in the chord-wise force
\end{enumerate} 

The resulting normal and chord-wise forces are shown in their raw form on Figure \ref{fig:CN_ramp_example} and \ref{fig:CC_ramp_example}. The dynamic stall angle based on the chordwise force $\alpha_{ds,CC}$ is shown by a black diamond in both figures. The experimental data contains a high level of noise around 50Hz. This frequency being close to the one of the alternating current, the noise is suspected to come from the AC-DC converter used by the motor. 

\begin{figure}[h]
	\begin{subfigure}{.5\textwidth}
		\includegraphics[width=\textwidth]{CN.png}
		\caption{Raw normal coefficient $C_N$}
		\label{fig:CN_ramp_example}
	\end{subfigure}%
	\begin{subfigure}{.5\textwidth}
		\includegraphics[width=\textwidth]{CC.png}
		\caption{Raw $C_N$}
		\label{fig:CC_ramp_example}
	\end{subfigure}
	\caption{Raw force signals as a function of convectime time, measured during a ramp-up experiment with $r=0.26$, $\alphadot = 10 \si{\degree \per \second}$}
\end{figure}

The experimental data was consequently filtered, retaking part of the procedure of Kenneth et al. \cite{kenneth_experiments_2011}. However, their procedure included a moving-average, a kind of low-pass filter that is known for lowering the amplitude of low frequencies. 
The filter performance was tested with a sinusoidal signal at $1 \si{\hertz}$ of peak-amplitude 1, on top of which noise was added. The goal of the test was to filter out the added noise without distording the original $1 \si{\hertz}$ signal. 
The noise consists of two $35 \si{\hertz}$ and $50 \si{\hertz}$ signals with peak-amplitude 1 and random phase-lag. The perfomance appears to be much better when the moving average step si excluded: indeed, the original and filtered signals superimpose perfectly in this case (Figure \ref{fig:filter_test_without})

\begin{figure}[h]
	\begin{subfigure}{.5\textwidth}
		\includegraphics[width=\textwidth]{1Hz.png}
		\caption{Pure sinusoidal signal at frequency $f=1 \si{\hertz}$}
		\label{fig:1Hz}
	\end{subfigure}%
	\begin{subfigure}{.5\textwidth}
		\includegraphics[width=\textwidth]{noisy_1Hz.png}
		\caption{Sum of $1\si{\hertz} + 35\si{\hertz} + 50\si{\hertz}$ signals}
		\label{fig:noisy_1Hz}
	\end{subfigure}
	\caption{Input signal for the filter test. The two highest frequencies simulate the effect of noise.}
\end{figure}

\begin{figure}[h]
	\begin{subfigure}{.5\textwidth}
		\includegraphics[width=\textwidth]{filter_test_with.png}
		\caption{With moving average}
		\label{fig:filter_test_with}
	\end{subfigure}%
	\begin{subfigure}{.5\textwidth}
		\includegraphics[width=\textwidth]{filter_test_without.png}
		\caption{Excluding the moving average step}
		\label{fig:filter_test_without}
	\end{subfigure}
	\caption{Results of the filter test: comparison between original signal without noise and filtered signal.}
\end{figure}

The filtering procedure without moving average is applied to the measured forces before applying the dynamic stall models (Figure \ref{fig:forces_filt}). The dynamic stall onset obtained by looking for the minimum in $C_C$ is displayed by a black diamond. 

\begin{figure}[h]
	\begin{subfigure}{.5\textwidth}
		\includegraphics[width=\textwidth]{CN_filt.png}
		\caption{Filtered $C_N$}
		\label{fig:CN_filt}
	\end{subfigure}%
	\begin{subfigure}{.5\textwidth}
		\includegraphics[width=\textwidth]{CC_filt.png}
		\caption{Filtered $C_C$}
		\label{fig:CC_filt}
	\end{subfigure}
	\caption{Filtered force signals measured on a ramp-up experiment with $r=0.26$, $\alphadot = 10 \si{\degree \per \second}$}
	\label{fig:forces_filt}
\end{figure}